\documentclass[11pt]{article}
\usepackage{amsmath,amssymb, amsthm, marvosym, permute, extsizes}
\usepackage{siunitx, graphicx, float, enumitem, adjustbox, hyperref, bm}
\usepackage{microtype, dsfont}
\usepackage[normalem]{ulem}
\usepackage[T1]{fontenc}
\usepackage[utf8]{inputenc}
\usepackage{lmodern}
\usepackage[T1]{fontenc}
\usepackage[a4paper,margin=2.5cm]{geometry}
\usepackage[icelandic]{babel}
\newcommand{\explain}[2]{\underbrace{#1}_\textrm{$#2$}}

\usepackage{minted}

\title{\vspace{-3cm}Heimadæmi 1\\ \vspace{0.4cm} \large Kjarn og öreindafræði \vspace{-0.3cm}}
\author{Emil Gauti Friðriksson}
\begin{document}
\maketitle
\section*{Dæmi 1}
Jafna (2.6) fyrir bindiorku tvívetniskjarna á bls. 15 í kennslubókinni gerir ráð fyrir að bindiorkan sé mun lægri en orkan sem fólgin er í kyrrstöðumassa kjarnans. Notið varðveislu orku og skriðþunga til að leiða út jöfnuna og áætlið skekkjuna í svarinu sem hún gefur miðað við nákvæma útreikninga.

\subsection*{Svar}
Byrjum á að rita upp jöfnu(2.6)
\begin{align*}
    B = (M_n + M_{^1H}-M_{^2H})\cdot c^2 = E_\gamma + \frac{E_\gamma ^2}{2M_{^2H}c^2} = 2.225 MeV
\end{align*}

\noindent Við vitum að eftirfarandi er satt:
\begin{align*}
    M_n c^2 + M_{^1H} c^2   &= B + M_{^2H}c^2\\
    M_{^2H}c^2 + B          &= E_\gamma + \sqrt{M_{^2H}^2c^4 + p^2c^2}\\
                            &= E_\gamma + M_{^2H}c^2\sqrt{1 + \frac{p^2c^2}{M_{^2H}^2c^4}} \\
\text{(Taylor-liðun)}       &\approx E_\gamma + M_{^2H}c^2\left(1 + \frac{p^2c^2}{2M_{^2H}^2c^4}\right)\\
                            &=E_\gamma +  M_{^2H}c^2 + \frac{p^2c^2}{2M_{^2H}c^2}\\
                            &=E_\gamma +  M_{^2H}c^2 + \frac{E_\gamma}{2M_{^2H}c^2}\\
\Rightarrow             B   &=E_\gamma + \frac{E_\gamma}{2M_{^2H}c^2}
\end{align*}
Athugum að skriðþungi ljóseindarinnar og skriðþungi $M_{^2H}$ hlýtur að vera sá sami vegna varðveislu skriðþungans og því gildir $E_\gamma = pc$\\
Skekkjan í svarinu kemur frá Taylor-nálguninni en hún er $O(p^4)$.


\section*{Dæmi 2}
Notið massaformúlu Weizsäckers til að leysa eftirfarandi verkefni.
\begin{enumerate}[label=(\alph*)]
    \item Hvaða sætistala Z gefur stöðugasta kjarna með tiltekna oddstæða massatölu A? Notið niðurstöðuna til að finna sætistölu stöðugasta kjarnans með A = 111.
    \item Leiðið út eftirfarandi reglu um bindiorku fyrir samsætur með háa massatölu ef gert er ráð fyrir A sé oddstæð og Z jafnstæð: $$\frac { 1 } { 2 } ( B ( Z , A + 1 ) + B ( Z , A - 1 ) ) - B ( Z , A ) = \frac { \delta _ { 0 } } { A ^ { 1 / 2 } } + \ldots$$ þar sem $\delta_0 = 11.2$ MeV/$c^2$ og $\ldots$ stendur fyrir liði sem falla hraðar með vaxandi A.
\end{enumerate}


\subsection*{Svar(a)}
Athugum að M(Z,A) má reikna sem($\delta = 0$):
\begin{align*}
    M(Z,A) &= \alpha A - \beta Z + \gamma Z^2 + \delta\\
    \text{Þar sem:   } \alpha  &= M_n -a_v +a_sA^{-1/3} + \frac{a_a}{4}\\
                    \beta   &= a_a + (M_n - M_p - m_e)\\
                    \gamma  &= \frac{a_a}{A} + \frac{a_c}{A^{1/3}}\\
                    \delta  &= 0
\end{align*}
Athugum að bindiorkuna má reikna sem:
\begin{align*}
    B(Z,A)                                      &= [ZM(^1H) + (A-Z)M_n - M(A,Z)]c^2\\
                                                &= [Z(M_p + m_e) + (A-Z)M_n - (\alpha A - \beta Z + \gamma Z^2)]c^2\\
\text{diffrum: }\frac{\delta}{\delta Z}B(A,Z)   &=[M_p + m_e -M_n + \beta - 2\gamma Z]c^2\\
                                                &=\left[M_p + m_e -M_n + a_a + M_n - M_p - m_e - 2\left(\frac{a_a}{A}+\frac{a_c}{A^{1/3}} \right)Z\right]c^2\\
                                                &=\left[-2\left(\frac{a_a}{A}+\frac{a_c}{A^{1/3}} \right)Z+a_a\right]c^2\\
                                \Rightarrow Z   &= \frac{a_a}{2\left(\frac{a_a}{A}+\frac{a_c}{A^{1/3}} \right)} \\
                                                &= 47.15
\end{align*}
Hér fyrir ofan fundum við núllstöð $\frac{d}{dz} B(Z,A)$ í Z sem er hápunktur B(Z,A) og þar með stöðugasta ástandið sem fall af Z.












\section*{Dæmi 3}
Kjarninn $^27$Si er $\beta^+$ geislavirkur og hámarksorka jáeinda sem myndast við hrörnunina er 3.8 MeV. Notið þessar upplýsingar til að áætla gildi fastans $a_c$ í Coulomb-lið massaformúlu Weizsäckers. Berið svarið saman við gildið sem gefið er upp á bls. 17 í kennslubókinni. Ábending: Byrjið á að kanna hvaða liðir í massaformúlunni eru hinir sömu fyrir móður- og dótturkjarnann.

\subsection*{Svar}
Við athugum að M(A,Z)>M(A,Z-1) og þessi mismunur á orku ætti að vera jafn orku jáeindarinnar. Til einföldunar þá notum við eftirfarandi:
\begin{align*}
    M(A,Z) &= \alpha A - \beta Z + \gamma Z^2 + \frac{\delta}{A^{1/2}}\\
    \text{Þar sem:   } \alpha  &= M_n -a_v +a_sA^{-1/3} + \frac{a_a}{4}\\
                    \beta   &= a_a + (M_n - M_p - m_e)\\
                    \gamma  &= \frac{a_a}{A} + \frac{a_c}{A^{1/3}}\\
                    \delta  &= 0
\end{align*}
þá reiknum við:
\begin{align*}
   3.8MeV   &= M(A,Z)-M(A,Z-1) \\
            &=-\beta + 2\gamma Z - \gamma\\
            &= -a_a - (M_n - M_p - m_e) +\explain{(2Z-1)}{=27}\gamma\\
            &= -a_a - (M_n - M_p - m_e) + 27\left(\frac{a_a}{A=27}+\frac{a_c}{A^{1/3}=3}\right)\\
            &= 9a_c \explain{- M_n + M_p + m_e}{= -0.783 MeV}\\
            \Rightarrow a_c &= 0.509 MeV
\end{align*}
Þetta svar er ekki fjarri lagi gildinu sem gefið er í bókinni (0.714MeV)

\section*{Dæmi 4}

\end{document}
